\section{Desarrollo/Análisis}
En esta sección, se hablará sobre las conexiones realizadas en el mcu y los diagramas de las principales funciones que componen el circuito de la lavadora.\par
Lo más sencillo fue declarar todos los pines B como salidas, porque es donde van conectados los Bcds en cascada con resistencias y los displays. Para declarar esos pines como salidas basta con escribir \texttt{DDRB = 0xff}. Ahora, con respecto a los botones, son partes del circuito que se encargarán de causar las interrupciones para hacer funcionar el circuito, en cuanto a la carga baja, media y alta, se conectaron a los pines D1, D2 y D3 respectivamente. A cada una de ellas fue necesario habilitar su comando de interrupción adecuado. Revisando la hoja del fabricante, para la carga baja, se debe habilitar el \texttt{PCMSK2} con el macro \texttt{PCINT12} y el \texttt{GIMSK} con el macro \texttt{PCIE2} para surtir efecto en este pin. En la carga media y alta, por medio de \texttt{GIMSK} se debe habilitar la interrupción con la macro \texttt{INTO} y \texttt{INT1}, esto hace que los botones trabajen apropiadamente. Ahora, en cuanto al \texttt{ISR} para cada uno de estos pines corresponden a las macros \texttt{PCINT2\_vect}, \texttt{INT0\_vect} y \texttt{INT1\_vect}. Dentro de estas interrupciones, hay una variable con el nombre: \texttt{tiempo\_inicio} que se iguala al tiempo total para cada carga, donde previamente se realizó un \texttt{\#define} con dichas magnitudes, además, se hace un llamado a la función \texttt{led\_carga} que se representa por el siguiente diagrama:
\input{sch2.tex}
El diagrama anterior permite encender los LEDs con respecto al tiempo ya asignado en el enunciado del laboratorio, note que dentro de esta función hay un llamado a otra función: \texttt{estados}, esta es la máquina de estados para realizar la transición del encendido de los LEDs dependiendo si está en suministro, lavar, enjuague o centrifugar. El siguiente diagrama de flujo resume el comportamiento de esta función.
\input{sch3.tex}
\input{sch3.1.tex}
Esta máquina de estados se encarga de administrar los tiempos para cada carga, por lo que se define un intervalo de tiempo para cada etapa, esto con base al tiempo ya establecido. Entonces, una vez que se cumple cada desigualdad, se apaga el LED actual y se enciende el siguiente LED con el estado correspondiente. Esto para cada situación, dependiendo de la carga que se le asigne a la lavadora. Ahora, hay dos detalles en esta máquina de estados, lo primero es que para referescar los LEDs se hace uso de la interrupción \texttt{TIMER0\_COMPA\_vect} y esto con base al tiempo asignado, esto permite también llevar una cuenta de los segundos que pasan. Claramente hay que hacer las configuraciones para lograr esto, lo primero es con  \texttt{TCCR0A} y la macro \texttt{WGM12}, luego la variable \texttt{OCR1A} para obtener un 1 segundo, el prescaler de 256 y habilitar la interrupción. Esta interrupción se compone de una función: \texttt{display}, la cual es encargada de mostrar la cuenta regresiva de los números mostrados en los displays, el diagrama de flujo de esta función se muestra a continuación,
\input{sch4.tex}
de donde primero se define la varible como la resta de esta misma variable con segundos, luego con ayuda de condicional se comprueba que el tiempo de carga no sea 0 para poder separar unidades y decenas del número, así, se toman los 4 bits LSB y se colocan en las unidades, mientras que los 4 bits MSB en las decenas, se muestran los ceros y finalmente se reinician las variables a 0.


Lo segundo es que se muestra la función: \texttt{led\_off} que se encarga de limpiar los puertos donde están conectados las cargas y los estados, en otras palabras apaga los LEDs que indican el comportamiento actual de la lavadora.