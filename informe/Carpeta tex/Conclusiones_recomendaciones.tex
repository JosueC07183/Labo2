\section{Conclusiones y recomendaciones}
Para concluir, en este trabajo fue posible entender el comportamiento de un MCU por medio del uso de interrupciones, lo que es una manera avanzada para controlar comportamientos de otros dispositivos electrónicos, por ejemplo: LEDs y displays de 7 segmentos. El haber realizado una lectura detenida de la hoja del fabricante del MCU y luego realizar pruebas excesivas para comprender el funcionamiento de cada pin fue algo muy importante para darle sentido a lo solicitado en este trabajo. La falta de literatura disponible hizo que se realizaran muchos experimentos de acuerdo a la necesidad de este laboratorio, no se escatimó el tiempo dedicado a las muchas depuraciones que se realizaron, no obstante, fueron valiosas y poder tener una simulación adecuada para esta lavadora, pues por falta de tiempo no se logró la optimización esperado en el uso de los pines, es decir, hicieron falta para poder implementar el botón de inicio y pausa, por lo que la lavadora automática está programada de tal que seleccionar la carga ésta inicia su trabajo y se detiene hasta cuando termine el tiempo asignado para cada etapa de lavado.\par
Aún así, el trabajo realizado es satisfactorio en cuanto a funcionamiento porque las interrupciones funcionaron correctamente, es decir, los LEDs responden a ellas, los temporizadores respetan los tiempos de ejecución, en ese sentido, el trabajo fue impecable. 
Como recomendaciones para este trabajo:
\begin{itemize}
\item Hacer todas las depuraciones posibles para entender el funcionamiento de lo que se está realizando.
\item Revisar las configuraciones de las interrupciones, porque es posible que se olvidé realizar alguna habilitación y esto hace que se obtenga el resultado no esperado, y el compilador tampoco se lo va a decir. En ese sentido, revisar muy bien que botones son los que se usarán para interrupciones y cuales serán simplemente salidas. 
\end{itemize}